Enterprise resource planning (ERP) systems are comprehensive software
systems used to manage daily activities in enterprises. Such
activities include---but are not limited to---financial management
(accounting), production planning, supply chain management and
customer relationship management. ERP systems emerged as a remedy to
heterogeneous systems, in which data and functionality are spread
out---and duplicated---amongst dedicated subsystems. Instead, an ERP
system it built around a central database, which stores all
information in one place.

Traditional ERP systems such as
Microsoft Dynamics
NAV\footnote{\url{http://www.microsoft.com/en-us/dynamics/products/nav-overview.aspx}.},
Microsoft Dynamics
AX\footnote{\url{http://www.microsoft.com/en-us/dynamics/products/ax-overview.aspx}.}, and
SAP\footnote{\url{http://www.sap.com}.}
are three-tier architectures with a client, an application
server, and a centralised relational database system. The central database
stores information in tables, and the application server provides the
business logic, typically coded in a general purpose, imperative
programming language. A shortcoming to this approach is
that the state of the system is decoupled from the business logic,
which means that business processes---that is, the daily
activities---are not represented explicitly in the system. Rather,
business processes are encoded implicitly as transition systems, where
the state is maintained by tables in the database, and transitions are
encoded in the application server, possibly spread out across several
different logical modules.

The process-oriented event-driven transaction systems (POETS)
architecture introduced by Henglein et al.~\cite{henglein09jlap} is a
qualitatively different approach to ERP systems. Rather than storing
both transactional data and implicit process state in a database,
POETS employs a pragmatic separation between transactional data, which
is persisted in an \emph{event log}, and \emph{contracts}, which are
explicit representations of business processes, stored in a separate
module. Moreover, rather than using general purpose programming
languages to specify business processes, POETS utilises a declarative
domain-specific language (DSL)~\cite{andersen06sttt}. The use of a DSL
not only enables explicit formalisation of business processes, it also
minimises the gap between requirements and a running system. In fact,
Henglein et al. take it as a goal of POETS that ``[...] the formalized
requirements \emph{are} the system''  \cite[page 382]{henglein09jlap}.

The bird's-eye view of the POETS architecture is presented in
Figure~\ref{fig:origArch}.
\begin{figure}[t]
  \centering\small
  \newcommand\drawInner[2]{
  \node[anchor=east,align=center] (t) at (#2.east) {#1};%
  \draw[thick] (t.north west) -- (t.north east)%
            (t.south west) -- (t.south east);%
  }
\newcommand\drawHeader[2]{
  \node[anchor=north] at (#2.north) {#1};
}
\newcommand\drawList[2]{
  \node[fill=white,draw,align=left] at (#2.south) {#1};
}

\begin{tikzpicture}
  [box/.style={shape=rectangle,rounded corners=3mm,minimum height=3cm,
    minimum width=3.5cm, anchor=north,draw},%
  db/.style={very thick,shape=cylinder, shape border
    rotate=90,aspect=.25,draw,align=center},%
  node distance=8cm]%
  \node[thick,box] (contract) {};%
  \drawHeader{Contract engine}{contract}%
  \drawInner{Running\\contracts}{contract}%
  \drawList{start contract\\register event\\end contract}{contract}%

  \node[thick,box,right of=contract] (report) {};%
  \drawHeader{Report engine}{report}%
  \drawInner{Report\\definitions}{report}%
  \drawList{add report\\delete report\\get report\\query report}{report}%

  \node[db,anchor=north] (events) at ($(contract)!.5!(report)-(0,1)$) {Event\\log};%

  \draw[-latex] (contract) edge[]
  node[above,midway,sloped] {events} (events)%
  (events) edge[] node[above,midway,sloped]
  {updates} (report)%
  (report.150) -- node[above,midway] {query results} (contract.30);%
\end{tikzpicture}
%%% Local Variables: 
%%% mode: latex
%%% TeX-master: "tr"
%%% End: 

  \caption{Bird's-eye view of the POETS
    architecture (diagram copied from~\cite{henglein09jlap}).}
  \label{fig:origArch}
\end{figure}
At the heart of the system is the event log, which is an append-only
list of transactions. Transactions represent ``things that take
place'' such as a payment by a customer, a delivery of goods by a
shipping agency, or a movement of items in an inventory. The
append-only restriction serves two purposes. First, it is a legal
requirement in ERP systems that transactions, which are relevant for
auditing, are retained. Second, the report engine utilises
monotonicity of the event log for optimisation, as shown by Nissen and
Larsen~\cite{nissen07ifl}.

Whereas the event log stores historical data, contracts play the role
of describing which events are expected in the future. For instance, a
yearly payment of value-added tax (VAT) to the tax authorities is an
example of a (recurring) business process. The amount to be paid to
the tax authorities depends, of course, on the financial transactions
that have taken place. Therefore, information has to be derived from
previous transactions in the event log, which is realised as a
\emph{report}. A report provides structured data derived from the
transactions in the event log. Like contracts, reports are written in
a declarative domain-specific language---not only in order to minimise
the semantic gap between requirements and the running system, but also
in order to perform automatic optimisations.

Besides the radically different software architecture, POETS
distinguishes itself from existing ERP systems by abandoning
the double-entry bookkeeping (DEB) accounting
principle~\cite{weygandt04book} in favour of the resources, events,
and agents (REA) accounting model of McCarthy~\cite{mccarthy82tar}.

In double-entry bookkeeping, each transaction is recorded as two
postings in a \emph{ledger}---a \emph{debit} and a
\emph{credit}. When, for instance, a customer pays an amount $x$ to
a company, then a debit of $x$ is posted in a cash account, and a
credit of $x$ is posted in a sales account, which reflects the flow of
cash from the customer to the company. The central invariant of DEB
is that the total credit equals the total debit---if not,
resources have either vanished or spontaneously appeared. DEB fits
naturally in the relational database oriented architectures, since
each ledger is similar in structure to a table. Moreover, DEB is the
de facto standard accounting method, and therefore used by
current ERP systems.

In REA, transactions are not registered in accounts, but rather as the
events that take place. An event in REA is of the form $(a_1,a_2,r)$
meaning that agent $a_1$ transfers resource $r$ to agent $a_2$. Hence,
when a customer pays an amount $x$ to a company, then it is
represented by a single event
$(\mbox{customer},\mbox{company},x)$. Since events are atomic, REA
does not have the same redundancy\footnote{In traditional DEB,
  redundancy is a feature to check for consistency. However,
  in a computer system such redundancy is superfluous.} as DEB, and
inconsistency is not a possibility: resources always have an origin and a
destination.  The POETS architecture not only fits with the REA
ontology, it is based on it. Events are stored as first-class objects
in the event log, and contracts describe the expected future flow of
resources.\footnote{Structured contracts are not part of the original
  REA ontology but instead due to Andersen et
  al.~\cite{andersen06sttt}.} Reports enable computation of derived 
information that is inherent in DEB, and which may be a legal
requirement for auditing. For instance, a sales account---which
summarises (pending) sales payments---can be reconstructed from
information about initiated sales and payments made by customers. Such
a computation will yield the same \emph{derived} information as in
DEB, and the principles of DEB consistency will be fulfilled simply by
construction.

\subsection{Outline and Contributions}
The motivation for our work is to assess the POETS architecture in
terms of a prototype implementation. During the implementation process
we have added 
features to the architecture that were painfully missing. Moreover, in
the process we found that the architecture need not be tailored to the
REA ontology---indeed to ERP systems---but the applicability of our
generalised architecture to other domains remains future research.
Our contributions are as follows:
\begin{itemize}
\item We present a generalised and extended POETS architecture
  (Section~\ref{sec:extended-poets}) that has been fully
  implemented.
\item We present domain-specific languages (DSLs) for data modelling
  (Section~\ref{sec:data-model}), report specification
  (Section~\ref{sec:report-engine}), and contract specification
  (Section~\ref{sec:contract-engine}).
\item We demonstrate how to implement a small use case, from scratch,
  in our implemented system (Section~\ref{sec:use-case}). We provide
  the complete specification of the system, which demonstrates both
  the conciseness and domain-orientation\footnote{Compare the motto:
    ``[...] the formalized requirements \emph{are} the system''
    \cite[page 382]{henglein09jlap}.} of our approach. We conclude that
  the extended architecture is indeed well-suited for implementing ERP
  systems---although the DSLs and the data model may require
  additions for larger systems. Most notably, the amount of code
  needed to implement the system is but a fraction of what would be
  have to be implemented in a standard ERP system.
\item We describe how we have utilised state-of-the art software
  development tools in our implementation, especially how the tightly
  coupled DSLs are implemented
  (Section~\ref{sec:implementation-aspects}).
\end{itemize}

%%% Local Variables: 
%%% mode: latex
%%% TeX-master: "tr"
%%% End: 