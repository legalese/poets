\subsection{Types, Type Constraints and Type Schemes}
\label{sec:type-system}

The following grammar describes the type expressions that are used in
the report language:
\begin{center}
  \begin{bnf}
    \tau &\ebnf& r \bnfsep \alpha \bnfsep \tbool \bnfsep
    \tint \bnfsep \treal \bnfsep \tchar \bnfsep
    \ttimestamp \bnfsep \tduration
    &\\ &\bnfsep &
    \tdt \bnfsep \tlist{\tau} \bnfsep \tent{r} \bnfsep \tau_1 \to \tau_2 \bnfsep \tau_1
    + \tau_2 \bnfsep (\tau_1, \tau_2) \bnfsep ()
  \end{bnf}
\end{center}
where $r$ ranges over record names and $\alpha$ over type
variables.

The report language is polymorphically typed and permits to put
constraints on types, for example, subtyping constraints.  The
language of type constraints is defined as follows:
\begin{center}
  \begin{bnf}
    C &\ebnf& \tau_1 <: \tau_2 \bnfsep \fieldC{\tau_1}{f}{\tau_2}
    \bnfsep \eqC{\tau} \bnfsep \ordC{\tau}
  \end{bnf}
\end{center}

Intuitively, these constraints can be interpreted as follows:
\begin{itemize}
\item A \emph{subtype constraint} of the form $\tau_1 <: \tau_2$
  requires $\tau_1$ to be a subtype of $\tau_2$,
\item a \emph{field constraint} of the form
  $\fieldC{\tau_1}{f}{\tau_2}$ requires $\tau_1$ to be a record
  type containing a field $f$ of type $\tau_2$,
\item an \emph{equality constraint} of the form $\eqC{\tau}$ requires
  the type $\tau$ to have an equality predicate \prt{==} defined on it,
  and
\item an \emph{order constraint} of the form $\ordC{\tau}$ requires
  the type $\tau$ to have order predicates ($<$, \prt{<=}) defined on
  it.
\end{itemize}

In order to accommodate for the polymorphic typing, we have to move
from types to \emph{type schemes}. Type schemes are of the form
$\forall \tup \alpha. \tup C \implC \tau$, that is, a type with a
universal quantification over a sequence of type variables, restricted
by a sequence of constraints. We abbreviate $\forall \emptyseq. \tup C
\implC \tau$ by writing $\tup C \implC \tau$, and $\emptyseq \implC
\tau$ by $\tau$. The \emph{universal closure} of a type scheme $\tup C
\implC \tau$, that is, $\forall \tup \alpha. \tup C \implC \tau$ for
$\tup \alpha$ the free variables $\fv{\tup C,\tau}$ in $\tup C$ and
$\tau$, is abbreviated by $\forall \tup C \implC \tau$.

\subsection{Built-in Symbols}
\label{sec:built-symbols}


In the following we give an overview of the constants provided by the
language. Along with each constant $c$ we will associate a designated
type scheme $\sigma_c$.

One part of the set of constants consists of literals: Numeric
literals $\reals$, Boolean literals $\set{\ctrue, \cfalse}$, character
literals $\set{\texttt{'a'}, \texttt{'b'}, \ldots}$, and string
literals. Each literal is associated with its obvious type: $\tint$
(respectively $\treal$), $\tbool$, $\tchar$, respectively
$\tstring$. Moreover, we also have entity values $\valref{r,e}$ of
type $\tent{r}$ with $e$ a unique identifier.


In the following we list the remaining built-in constants along with
their respective type schemes. Many of the given constant symbols are
used as mixfix operators. This is indicated by placeholders
$\mhole$. For example a binary infix operator $\circ$ is then written
as a constant $\mhole \circ \mhole$. For a constant $c$ we write $c :
\tup C \implC \tau$ in order to indicate the type scheme $\sigma_c =
\forall \tup C \implC \tau$ assigned to $c$.

\begin{align*}
  \mhole\circ\mhole &\colon \alpha <: \treal \implC \alpha
  \to \alpha \to \alpha &&\forall
  \circ \in \set{+,-,*}\\
  \mhole / \mhole&\colon \treal \to \treal \to
  \treal \\
  \mhole \equiv \mhole &\colon \eqC{\alpha} \implC \alpha \to
  \alpha \to \tbool\\
  \mhole\circ\mhole &\colon \ordC{\alpha} \implC \alpha \to
  \alpha \to \tbool &&\forall
  \circ \in \set{>,\ge,<,\le}\\
  \mhole\circ\mhole &\colon \alpha <: \tdt \implC \alpha
  \to \tduration \to \alpha &&\forall
  \circ \in \set{\dplus,\dminus}
\end{align*}
%
\begin{align*}
  r\;\{f_1 = \mhole, \ldots , f_n
    = \mhole\} &\colon \tau_1 \to \ldots \tau_n \to r \qquad
  \text{where } \rho(r) = \set{(f_1,\tau_1),\ldots,(f_n,\tau_n)}\\
  \mhole . f &\colon \fieldC{\alpha}{f}{\beta} \implC
  \alpha \to \beta\\
  \mhole\;\{f_1 = \mhole, \ldots , f_n
    = \mhole\} &\colon
  \fieldC{\alpha}{f_1}{\alpha_1},\ldots,\fieldC{\alpha}{f_n}{\alpha_n}
  \implC \alpha \to \alpha_1 \to \ldots \to \alpha_n \to \alpha
\end{align*}
%
\begin{align*}
  \lnot &: \tbool \to \tbool\\
  \mhole \circ \mhole &: \tbool \to \tbool \to \tbool &&\forall
  \circ \in \set{ \land, \lor }\\
  \ifelse{\mhole}{\mhole}{\mhole} &\colon \tbool \to \alpha \to \alpha \to \alpha
\end{align*}
%
\begin{align*}
  \vlist{}&\colon \tlist{\alpha}\\
  \mhole \cons\mhole &\colon \alpha \to \tlist{\alpha} \to
  \vlist{\alpha}\\
  \foldr &\colon \eqC\beta \implC (\alpha \to \beta \to \beta) \to \beta
  \to \tlist\alpha \to \beta
\end{align*}
%
\begin{align*}
  () &\colon ()\\
  (\mhole,\mhole)&\colon \alpha \to \beta \to
  (\alpha,\beta)\\
  \keyword{Inl}&\colon \alpha \to \alpha + \beta\\
  \keyword{Inr}&\colon \beta \to \alpha + \beta\\
  \keyword{case}&\colon \alpha + \beta \to (\alpha \to
    \gamma) \to (\beta \to \gamma) \to \gamma\\
  \mhole.1&\colon (\alpha,\beta)\to\alpha\\
  \mhole.2&\colon (\alpha,\beta)\to\beta
\end{align*}
%
\begin{align*}
  \mhole\derefNow&\colon \tent{r} \to r\\
  \mhole\derefCxt&\colon \tent{r} \to r
\end{align*}
%
\begin{align*}
  \vdate{\mhole-\mhole-\mhole\quad\mhole\colon\mhole\colon\mhole}
  &: \underbrace{\tint \to \dots \to \tint}_{6\times} \to \ttimestamp%
  \\%
  \vdur{\mhole\; s, \mhole\; min, \mhole\;
    h,  \mhole\; d,  \mhole\;
    w,  \mhole\; mon, \mhole\; y}
  &: \underbrace{\tint \to \dots \to \tint}_{7\times} \to \tduration%
\end{align*}
%
\begin{align*}
  \keyword{error} &: \tstring \to \alpha
\end{align*}

We assume that there is always defined a record type $\recordname{Event}$
which is the type of an event stored in the central \emph{event log}
of the system. The list of all events in the event log can be accessed
by the following constant:
\begin{align*}
  \keyword{events} &: \tlist{\recordname{Event}}
\end{align*}

When considering built-in constants, we also distinguish between
\emph{defined functions} $f$ and \emph{constructors}
$F$. Constructors are the constants $\vdate{\mhole-\mhole-\mhole\quad\mhole\colon\mhole\colon\mhole}$,
$\vdur{\mhole\; s, \mhole\; min, \mhole\; h, \mhole\; d, \mhole\; w,
  \mhole\; mon, \mhole \; y}$, $r\;\{f_1 = \mhole, \ldots ,
f_n = \mhole\}$, $\cons$, $\vlist{}$, $()$, $(\mhole,\mhole)$,
$\keyword{Inl}$, $\keyword{Inr}$ and $\keyword{error}$ as well as all
literals. The remaining constants are defined functions.

Derived from its type scheme we can also assign an \emph{arity}
$\arity{c}$ to each constant $c$ by defining $\arity{c}$ as the
largest $n$ such that $\sigma_c = \forall \tup\alpha.\tup C \implC
\tau_1 \to \tau_2 \to \dots \to \tau_{n+1}$

\subsection{Type System}
\label{sec:constraint-system}

Before we can present the type system of the report language, we have
to give the rules for the type constraints. To this end we extend the
subtyping judgement $\issubtype{\calR}{\tau_1}{\tau_2}$ for values
from Figure~\ref{fig:value-typing}. The constraint entailment
judgement $\constrEnt{\calR}{\calC}{C}$ states that a constraint $C$
follows from the set of constraints $\calC$ and the record typing
environment $\calR$.
\begin{figure}
  \centering%
  \small%
  % 
  \infruleI{Hyp}%
  {C \in \calC}%
  {\constrEnt{\calR}{\calC}{C}}%
  \quad%
  \infruleI{$<:$ Rec}
  {r_1 \subrec r_2}%
  {\constrEnt{(R,A,F,\rho,\subrec)}{\calC}{r_1 <: r_2}}%
  \\[1em]% 
  \infrule{$<:$ Refl}%
  {\constrEnt{\calR}{\calC}{\tau <: \tau}}%
  \quad%
  \infruleII{$<:$ Trans}%
  {\constrEnt{\calR}{\calC}{\tau_1 <: \tau_2}}
  {\constrEnt{\calR}{\calC}{\tau_2 <: \tau_3}}
  {\constrEnt{\calR}{\calC}{\tau_1 <: \tau_3}}
  \\[1em]%
  \infruleII{$<:$ Fun}
  {\constrEnt\calR\calC{\tau_1 <: \tau_2}}
  {\constrEnt\calR\calC{\tau_3 <: \tau_4}}
  {\constrEnt\calR\calC{\tau_2 \to \tau_3 <: \tau_1 \to \tau_4}}
  \quad%
  \infruleI{$<:$ List}
  {\constrEnt\calR\calC{\tau_1 <: \tau_2}}
  {\constrEnt\calR\calC{\tlist{\tau_1} <: \tlist{\tau_2}}}
  \\[1em]%
  \infruleII{$<:$ Sum}
  {\constrEnt\calR\calC{\tau_1 <: \tau_2}}
  {\constrEnt\calR\calC{\tau_3 <: \tau_4}}
  {\constrEnt\calR\calC{\tau_1 + \tau_3 <: \tau_2 + \tau_4}}
  \\[1em]%
  \infruleII{$<:$ Prod}
  {\constrEnt\calR\calC{\tau_1 <: \tau_2}}
  {\constrEnt\calR\calC{\tau_3 <: \tau_4}}
  {\constrEnt\calR\calC{(\tau_1, \tau_3) <: (\tau_2, \tau_4)}}
  \\[1em]%
  \infrule{$<:$ Num}
  {\constrEnt\calR\calC{\tint <: \treal}}
  \quad%
  \infrule{$<:$ Timestamp}
  {\constrEnt\calR\calC{\ttimestamp <: \tdt}}
  \\[1em]%
  \infrule{$<:$ Duration}
  {\constrEnt\calR\calC{\tduration <: \tdt}}
  \\[2em]
  \infruleI{Field}
  {(f,\tau) \in \rho(r)}
  {\constrEnt{(R,A,F,\rho,\subrec)}\calC{\fieldC{r}{f}{\tau}}}
  \\[1em]%
  \infruleII{Field Prop}%
  {\constrEnt\calR\calC{\fieldC{\tau_1}{f}{\tau_2}}}%
  {\constrEnt\calR\calC{\tau'_1 <: \tau_1}}%
  {\constrEnt\calR\calC{\fieldC{\tau'_1}{f}{\tau_2}}}
  \\[2em]%
  \infruleI{Ord Base}%
  {\tau \in \set{\tbool, \tint, \treal, \tchar, \tduration, \ttimestamp, \tdt}}%
  {\constrEnt\calR\calC{\ordC{\tau}}}%
  \\[1em]% 
  \infruleI{Eq Ord}%
  {\constrEnt\calR\calC{\ordC{\tau}}}%
  {\constrEnt\calR\calC{\eqC{\tau}}}%
  \quad%
  \infruleI{Eq Rec}%
  { r\in R}%
  {\constrEnt{(R,A,F,\rho,\subrec)}\calC{\eqC{r}}}%
  \\[2em]%
  \infruleIII{$P$ $F$}%
  {F \in \set{(\cdot,\cdot), +, \tlist{\cdot},\tent{\cdot}}}%
  {P \in \set{\ordC{\cdot},\eqC{\cdot}}}%
  {\forall 1 \le i \le n\colon\;\constrEnt\calR\calC{P(\tau_i)}}%
  {\constrEnt\calR\calC{P(F(\tau_1, \dots,\tau_n))}}%
  % 
  \caption{Type constraint entailment $\constrEnt{\calR}{\calC}{C}$.}
  \label{fig:constr}
\end{figure}

The type constraint entailment judgement $\constrEnt\calR\calC C$ is
straightforwardly extended to sequences of constraints $\tup C$. We
define that $\constrEnt\calR\calC C_1,\dots,C_n$ iff
$\constrEnt\calR\calC C_i$ for all $ 1\le i \le n$.


The type system of the report language is a straightforward
polymorphic lambda calculus extended with type constraints. The typing
judgement for the report language is written
$\reptype\calR{\calC}{\Gamma}{e}{\sigma}$, where $\calR$ is a record
typing environment, $\calC$ a set of type constraints, $\Gamma$ a type
environment, $e$ an expression and $\sigma$ a type scheme. The
inference rules for this judgement are given in
Figure~\ref{fig:reptype}.

\begin{figure}[t]
  \centering%
  \small%
  \infruleI{Var}%
  { x : \sigma \in \Gamma }%
  {\reptype\calR{\calC}{\Gamma}{x}{\sigma}}%
  \quad%
  \infrule{Const}%
  { \reptype\calR{\calC}{\Gamma}{c}{\sigma_c}}%
  \\[1em]%
  \infruleII{Sub}%
  {\reptype\calR{\calC}{\Gamma}{e}{\tau}}%
  {\constr{\calC}{\tau <: \tau'}} %
  {\reptype\calR{\calC}{\Gamma}{e}{\tau'}}%
  \quad%
  \infruleI{Abs}%
  {\reptype\calR{\calC}{\Gamma\cup\set{x:\tau}}{e}{\tau'}}%
  {\reptype\calR{\calC}{\Gamma}{\lambda x \to e}{\tau \to \tau'}}%
  \\[1em]%
  \infruleII{App}%
  {\reptype\calR{\calC}{\Gamma}{e_1}{\tau_1 \to \tau_2}}%
  {\reptype\calR{\calC}{\Gamma}{e_2}{\tau_1}}%
  {\reptype\calR{\calC}{\Gamma}{e_1 \;      e_2}{\tau_2}}%
  \\[1em]%
  \infruleII{Let}%
  {\reptype\calR{\calC}{\Gamma}{e_1}{\sigma}}%
  {\reptype\calR{\calC}{\Gamma \cup \set{x : \sigma}}{e_2}{\tau}}%
  {\reptype\calR{\calC}{\Gamma}{\letin{x = e_1}{e_2}}{\tau}}%
  \\[1em]%
  \infruleII{Type Of}%
  {\begin{gathered}[b]
      \reptype\calR{\calC}{\Gamma}{e}{r'}\\%
      \constrEnt\calR{\calC}{r <: r'}
    \end{gathered}}%
  {\begin{gathered}[b]
      \reptype\calR{\calC}{\Gamma \cup \set{x : r}}{e_1}{\tau}\\%
      \reptype\calR{\calC}{\Gamma \cup \set{x : r'}}{e_2}{\tau}
    \end{gathered}}%
  {\reptype\calR{\calC}{\Gamma}{\typeof{x = e}{\{r \to e_1
          \semi \default \to e_2\}}}{\tau}}%
  \\[1em]%
  \infruleII{Type Of Ref}%
  {\begin{gathered}[b]
      \reptype\calR{\calC}{\Gamma}{e}{\tent{r'}}\\%
      \constrEnt\calR{\calC}{r <: r'}
    \end{gathered}}%
  {\begin{gathered}[b]
      \reptype\calR{\calC}{\Gamma \cup \set{x : \tent{r}}}{e_1}{\tau}\\%
      \reptype\calR{\calC}{\Gamma \cup \set{x : \tent{r'}}}{e_2}{\tau}
    \end{gathered}}%
  {\reptype\calR{\calC}{\Gamma}{\typeof{x = e}{\{\tent{r} \to e_1
          \semi \default \to e_2\}}}{\tau}}%
  \\[1em]%
  \infruleII{$\forall$ Intro}%
  {\reptype\calR{\calC \cup \tup C}{\Gamma}{e}{\tau}}%
  {\tup\alpha \not\in \fv{\calC}\cup \fv{\Gamma}}%
  {\reptype\calR{\calC}{\Gamma}{e}{\forall\tup\alpha.\tup C\implC \tau}}%
  \\[1em]%
  \infruleII{$\forall$ Elim}%
  {\reptype\calR{\calC}{\Gamma}{e}{\forall\tup\alpha.\tup C \implC
      \tau'}}%
  {\constrEnt\calR{\calC}{\subst{\tup C}{\tup\alpha}{\tup\tau}}}%
  {\reptype\calR{\calC}{\Gamma}{e}{\subst{\tau'}{\tup\alpha}{\tup\tau}}}%
  %
  %
  \caption{Type inference rules for the report language.}
  \label{fig:reptype}
\end{figure}

A typing $\reptype{\calR}{\calC'}{\Gamma'}{e}{\tau'}$ is an instance of
$\reptype{\calR}{\calC}{\Gamma}{e}{\tau}$ iff there is a substitution $S$
such that $ \Gamma'\supseteq \Gamma S$, $\tau' = \tau S$, and
$\constrEnt\calR{\calC'}{\calC S}$. Deriving from that we say that the type
scheme $\sigma' = \forall \tup \alpha'. \tup C' \implC \tau'$ is an
instance of $\sigma = \forall \tup \alpha. \tup C \implC \tau$,
written $\sigma' < \sigma$, iff there is a substitution $S$ with
$\dom{S} = \alpha$ such that $\tau' = \tau S$ and
$\constrEnt\calR{\calC'}{\calC S}$.

Top-level function definitions are of the form
\[
f \; x_1 \; \dots \; x_n = e
\]
and can be preceded by an explicit type signature declaration of the
form $f : \sigma$.

Depending on whether an explicit type signature is present, the
following inference rules define the typing of top-level function definitions:
\begin{center}
  \infruleII{Fun}%
  {\reptype\calR{\calC \cup \tup
      C}{\Gamma\cup\set{x_1:\tau_1, \dots,
        x_n:\tau_n}}{e}{\tau}}%
  {\tup\alpha \not\in \fv{\calC}\cup \fv{\Gamma}}%
  {\reptype\calR{\calC}{\Gamma}{f \; x_1 \; \dots \; x_n =
      e}{\forall\tup\alpha.\tup C \implC \tau_1 \to \dots \to \tau_n
      \to \tau}}%
  \\[1em]
  \infruleII{Fun'}%
  {\reptype\calR{\calC}{\Gamma}{f \; x_1 \; \dots \; x_n = e}{\sigma}}%
  {\sigma' < \sigma}%
  {\reptype\calR{\calC}{\Gamma}{f \colon \sigma' \semi f \; x_1 \; \dots \;
      x_n = e}{\sigma'}}%
\end{center}

\subsection{Operational Semantics}
\label{sec:oper-semant}

In order to simplify the presentation of the operational semantics we
assign to each constant $c$ of the language its set of \emph{strict
  argument positions} $\strict{c} \subseteq \set{1,\dots,\arity{c}}$:

\begin{align*}
  \strict{\mhole\circ\mhole} &= \set{1,2} && \text{ for all binary
    operators } \circ \neq \cons \\
  \strict{c} &= \set{1} && \forall c \in \set{\lnot,
    \ifelse\mhole\mhole\mhole,\keyword{case},\keyword{error}, \mhole\derefCxt,\mhole\derefNow}\\
  \strict{\mhole. f} &= \set{1}\\
  \strict{\mhole\;\{\tup{f_i = e_i}\}} &= \set{1}\\
  \strict{\mhole.i} &= \set{1}
\end{align*}
For all other constraints $c$ for which the above equations do not
apply $\strict{c}$ is defined as the empty set $\emptyset$.

\emph{Values} form a subset of expressions which are fully evaluated
at the top-level. Such expressions are also said to be in \emph{weak
  head normal form} (\emph{whnf}). An expression is in weak head
normal form, if it is an application of a built-in function to too few
arguments, an application of a constructor, or a lambda
abstraction. Moreover, if a value is not of the form
$\keyword{error}\; v$, it is called \emph{defined}:


\begin{center}
  \begin{align*}
    v \ebnf\ &c \; e_1 \dots \; e_n && n < \arity{f}%
    \\ \bnfsep &F \; e_1 \dots \; e_n && n = \arity{F}, \forall i \in
    \strict{F}\quad e_i \text{ is defined value} %
    \\ \bnfsep &\lambda x \to e %
  \end{align*}
\end{center}

An even more restricted subset of the set of values is the set of
\emph{constructor values} which are expressions in \emph{constructor
  head normal form}. It is similar to weak head normal form, but with
the additional restriction, that arguments of a fully applied
constructor are in constructor normal form as well:

\begin{align*}
  V \ebnf\ &c \; e_1 \dots \; e_n && n < \arity{f}%
  \\ \bnfsep &F \; V_1 \dots \; V_n && n = \arity{F}, \forall i \in
  \strict{F}\quad V_i \text{ is defined} %
  \\ \bnfsep &\lambda x \to e %
\end{align*}

To further simplify the presentation we introduce evaluation
contexts. The following evaluation context $\ecxt$ corresponds to weak
head normal forms:
\begin{align*}
  \ecxt \ebnf &\cxthole \bnfsep \ecxt \; e \bnfsep \typeof{x =
    \ecxt}{\{r \to e_1\semi \default \to e_2\}} \\ \bnfsep &c\;
  e_1 \; \dots e_{i-1} \; \ecxt \; e_{i+1} \; \dots \; e_n\qquad
  \begin{aligned}[t]
    &i\in \strict{c}, n = \arity{c},\\
    &\forall j < i, j\in \strict{c}\colon e_j \text{ is defined value}
  \end{aligned}
\end{align*}

The evaluation context $\fcxt$ corresponds to constructor head normal
forms:
\begin{align*}
  \fcxt \ebnf &\cxthole \bnfsep \ecxt \; e \bnfsep \typeof{x =
    \ecxt}{\{r \to e_1\semi \default \to e_2\}}%
  \\%
  \bnfsep &f\; e_1 \; \dots e_{i-1} \; \ecxt \; e_{i+1} \; \dots \;
  e_n\qquad
  \begin{aligned}[t]
    &i\in \strict{f}, n = \arity{f},\\
    &\forall j < i, j\in \strict{f}\colon e_j \text{ is defined value}
  \end{aligned}
  \\%
  \bnfsep &F\; V_1 \; \dots V_{i-1} \; \fcxt \; e_{i+1} \; \dots \;
  e_n\qquad n= \arity{F}, V_1 \; \dots V_{i-1} \text{ are defined}
\end{align*}

Computations take place in a context of an \emph{event log}, i.e.\ a
sequence of values of type $\recordname{Event}$. In the following
definition of the semantics of the report language we use
$(ev_i)_{i<n}$ to refer to this sequence, where each $ev_i$ is of the
form $r \{\tup{f_j = e_j}\}$ with $r \subrec \recordname{Event}$.

We assume that the \recordname{Event} record type has a field
\fieldname{internalTimeStamp} that records the time at which the event
was added to the log. For each $ev_i$, we define its extension $ev'_i$
as follows: Each occurrence of an entity value $\valref{r,e}$ is
replaced by $\valref{r,e,t}$ where $t$ is the value of the
\fieldname{internalTimeStamp} field of $ev_i$. This will allow us to
define the semantics of the contextual dereference operator
$\derefCxt$. The semantics of both the $\derefCxt$ and the $\derefNow$
operator are given by the $\entlookup$ operator, which is provided by
the entity store, compare Section~\ref{sec:entity-store}. In order to
retrieve the latest value associated to an entity, we assume the
timestamp $t_\mathrm{now}$ that denotes the current time.

The rules describing the semantics of the report language in the form
of a small step transition relation $\step$ are given in
Figure~\ref{fig:repSem}.
\begin{figure}[t]
  \centering%
  \small%
  \infruleI{Context}%
  {e \step e'}%
  {\fapp{e} \step \fapp{e'}}%
  \quad%
  \infrule{Error}%
  {\fapp{\keyword{error}\; v} \step \keyword{error}\; v} %
  \\[1em]%
  \infrule{Abs}%
  {(\lambda x \to e_1) e_2 \step \subst{e_1}{x}{e_2} }%
  \quad%
  \infrule{Let}%
  {\letin{x = e_1}{e_2} \step \subst{e_2}{x}{e_1} }%
  \\[1em]%
  \infruleII{Type suc}%
  {r' \subrec r}%
  {v = r'\{\dots\}}%
  {\typeof{x = v}{\{r \to e_1 \semi \default \to e_2\}} \step
    \subst{e_1}{x}{v}}%
  \\[1em]%
  \infruleII{Type def}%
  {r' \not\subrec r}%
  {v = r'\{\dots\}}%
  {\typeof{x = v}{\{r \to e_1 \semi \default \to e_2\}} \step
    \subst{e_2}{x}{v}}%
  \\[1em]%
  \infruleII{Mod}%
  {\begin{aligned}[t]
      &\text{injection } \phi\colon \set{1,\dots,m} \hookrightarrow \set{1,\dots,n}\\%
      &\forall j \in \set{1,\dots, m} \colon\;f'_j = f_{\phi(j)} 
    \end{aligned}}%
  {e''_i = \begin{cases} e'_{\phi^{-1}(i)} &\text{if
        } i \in \image{\phi}\\
        e_i &\text{otherwise}
      \end{cases} }%
  {r\{f_1 = e_1,\dots,f_n =
    e_n\}\;\{f'_1 = e'_1,\dots,f'_m =
    e'_m\} \step r\{f_1 = e''_1,\dots,f_n
    = e''_n\}}%
  \\[1em]%
  \infrule{Acc}%
    {r\{f_1 = e_1,\dots,f_n = e_n\}
      . f_i \step e_i}%
    \quad%
    \infrule{If True}%
    {\ifelse{\keyword{True}}{e_1}{e_2} \step e_1}%
    \\[1em]%
    \infrule{If False}%
    {\ifelse{\keyword{False}}{e_1}{e_2} \step e_2}%
    \quad%
    \infrule{Case Left}%
    {\keyword{case} \; (\keyword{Inl}\; e) \; e_1 \; e_2 \step e_1 \; e}%
    \\[1em]%
    \infrule{Case Right}%
    {\keyword{case} \; (\keyword{Inr}\; e) \; e_1 \; e_2 \step e_2 \; e}%
    \quad%
    \infruleI{Proj}%
    { i \in \set{1,2} }%
    { (e_1,e_2) . i \step e_i }%
    \\[1em]%
    \infrule{Events}%
    {\keyword{events} \step \vlist{ev_1, ev_2,\dots,ev_n}}%
    \quad%
    \infrule{Fold Empty}%
    { \foldr \; e_1 \; e_2 \; \vlist{} \step e_2 }%
    \\[1em]%
    \infrule{Fold Cons}%
    { \foldr \; e_1 \; e_2 \; (e_3 \cons e_4) \step e_1 \; e_3
      \; (\foldr \; e_1 \; e_2 \; e_4) }%
    \\[1em]
    \infruleI{$\derefNow$ ignore}%
    {\entlookup_{t_\mathrm{now}}(e,t_\mathrm{now}) = v}%
    {\valref{r,e,t}\derefNow \step v}%
    \quad%
    \infruleI{$\derefNow$}%
    {\entlookup_{t_\mathrm{now}}(e,t_\mathrm{now}) = v}%
    {\valref{r,e}\derefNow \step v}%
    \\[1em]
    \infruleI{$\derefCxt$}%
    {\entlookup_{t_\mathrm{now}}(e,t) = v}%
    {\valref{r,e,t}\derefCxt \step v}%
    \quad%
    \infruleI{$\derefCxt$ now}%
    {\entlookup_{t_\mathrm{now}}(e,t_\mathrm{now}) = v}%
    {\valref{r,e}\derefCxt \step v}%
\caption{Small step operational semantics of the report language.}
\label{fig:repSem}
\end{figure}

%%% Local Variables: 
%%% mode: latex
%%% TeX-master: "tr"
%%% End: 