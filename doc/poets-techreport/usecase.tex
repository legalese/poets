In this section we describe a use case instantiation of POETS, which
we refer to as \muerp. With \muerp we model a simple ERP system for a
small bicycle shop. Naturally, we do not intend to model all features
of a full-blown ERP system, but rather we demonstrate a limited set of
core ERP features. In our use case, the shop purchases bicycles from
a bicycle vendor, and sells those bicycles to customers. We want to
make sure that the bicycle shop only sells bicycles in stock, and we
want to model a repair guarantee, which entitles customers to
have their bikes repaired free of charge up until three months after purchase.

Following Henglein et al.~\cite{henglein09jlap}, we also provide core
financial reports, namely the income statement, the balance sheet, the
cash flow statement, the list of open (not yet paid) invoices, and the
value-added tax (VAT) report. These reports are typical, minimal legal
requirements for running a business. We provide some example code in
this section, and the complete specification is included in
Appendix~\ref{app:muerp}. As we have seen in
Section~\ref{sec:extended-poets}, instantiating POETS amounts to
defining a data model, a set of reports, and a set of contract
templates. We describe each of these components in the following
subsections.

\subsection{Data Model}

The data model of \muerp is tailored to the ERP domain in accordance
with the REA ontology~\cite{mccarthy82tar}. Therefore, the main
components of the data model are resources, transactions (that is, events
associated with contracts), and agents. The complete data model is
provided in Appendix~\ref{app:muerp-ontology}.

~

\noindent\textbf{Agents} are modelled as an abstract type
\recordname{Agent}. An agent is either a \recordname{Customer}, a
\recordname{Vendor}, or a special \recordname{Me}
agent. \recordname{Customer}s and \recordname{Vendor}s are equipped
with a name and an address. The \recordname{Me} type is used to
represent the bicycle company itself. In a more elaborate example, the
\recordname{Me} type will have subtypes such as \recordname{Inventory}
or \recordname{SalesPerson} to represent subdivisions of, or
individuals in, the company. The agent model is summarised below:
\begin{lstlisting}[language=ontology,basicstyle=\small,multicols=2]
Agent is Data.

Customer is an Agent.
Customer has a String called name.
Customer has an Address.

Me is an Agent.

Vendor is an Agent.
Vendor has a String called name.
Vendor has an Address.
\end{lstlisting}
\vspace{-12pt}
\noindent\textbf{Resources} are---like
agents---\recordname{Data}. In our modelling of resources, we make a
distinction between \emph{resource types} and \emph{resources}. A
resource type represents a kind of resource, and resource types are
divided into currencies (\recordname{Currency}) and item types
(\recordname{ItemType}). Since we are modelling a bicycle shop, the
only item type (for now) is bicycles (\recordname{Bicycle}). A
resource is an instance of a resource type, and---similar to resource
types---resources are divided into money (\recordname{Money}) and
items (\recordname{Item}). Our modelling of items assumes an implicit
unit of measure, that is we do not explicitly model units of measure
such as pieces, boxes, pallets, etc.  Our resource model is summarised
below:
\begin{lstlisting}[language=ontology,basicstyle=\small,multicols=2]
ResourceType is Data.
ResourceType is abstract.

Currency is a ResourceType.
Currency is abstract.

DKK is a Currency.
EUR is a Currency.

ItemType is a ResourceType.
ItemType is abstract.

Bicycle is an ItemType.
Bicycle has a String called model.

Resource is Data.
Resource is abstract.

Money is a Resource.
Money has a Currency.
Money has a Real called amount.

Item is a Resource.
Item has an ItemType.
Item has a Real called quantity.  
\end{lstlisting}

\noindent\textbf{Transactions} (events in the REA terminology)
are, not surprisingly, subtypes of the built-in
\recordname{Transaction} type. The only transactions we consider in
our use case are bilateral transactions (\recordname{BiTransaction}),
that is transactions that have a sender and a receiver. Both the
sender and the receiver are agent entities, that is a bilateral
transaction contains references to two agents rather than copies of
agent data. For our use case we model payments (\recordname{Payment}),
deliveries (\recordname{Delivery}), issuing of invoices
(\recordname{IssueInvoice}), requests for repair of a set of items
(\recordname{RequestRepair}), and repair of a set of items
(\recordname{Repair}). Issuing of invoices contain the relevant
information for modelling of VAT, encapsulated in the
\recordname{OrderLine} type. We include some of these definitions
below:
\begin{lstlisting}[language=ontology,basicstyle=\small,multicols=2]
BiTransaction is a Transaction.
BiTransaction is abstract.
BiTransaction has an Agent entity
  called sender.
BiTransaction has an Agent entity
  called receiver.

Transfer is a BiTransaction.
Transfer is abstract.

Payment is a Transfer.
Payment is abstract.
Payment has Money.

CashPayment is a Payment.
CreditCardPayment is a Payment.
BankTransfer is a Payment.

IssueInvoice is a BiTransaction.
IssueInvoice has a list of OrderLine
   called orderLines.
\end{lstlisting}

Besides agents, resources, and transactions, the data model defines
the output types of reports (Appendix~\ref{app:muerp-ontology-report})
the input types of contracts
(Appendix~\ref{app:muerp-ontology-contract}), and generic data
definitions such as \recordname{Address} and \recordname{OrderLine}.
 The report types define
the five mandatory reports mentioned earlier, and additional
\recordname{Inventory} and \recordname{TopNCustomers} report
types. The contract types define the two types of contracts for the
bicycle company, namely \recordname{Purchase} and \recordname{Sale}.

\subsection{Reports}

Report specifications are divided into prelude functions
(Appendix~\ref{app:muerp-reports-prelude}), domain-specific prelude
functions (Appendix~\ref{app:muerp-reports-dsprelude}), internal
reports (Appendix~\ref{app:muerp-reports-internal}), and external
reports (Appendix~\ref{app:muerp-reports-external}).

~

\noindent\textbf{Prelude functions} are utility functions that are
independent of the custom data model. These functions are
automatically added to all POETS instances, but they are included in
the appendix for completeness. The prelude includes standard functions
such as \prt{filter}, but it also includes generators for accessing
event log data such as \prt{reports}. The event log generators provide
access to data that has a lifecycle such as contracts or reports,
compare Section~\ref{sec:event-log}.

~

\noindent\textbf{Domain-specific prelude functions} are utility
functions that depend on the custom data model. The
\prt{itemsReceived} function, for example, computes a list of all
items that have been delivered to the company, and it hence relies on
the \recordname{Delivery} transaction type (\prt{normaliseItems} and
\prt{isMe} are also defined in
Appendix~\ref{app:muerp-reports-dsprelude}):
\begin{lstlisting}[language=parrot,basicstyle=\small]
itemsReceived : [Item]
itemsReceived = normaliseItems [is |
  tr <- transactionEvents,
  del : Delivery = tr.transaction,
  not(isMe del.sender) && isMe del.receiver,
  is <- del.items]
\end{lstlisting}

~

\noindent\textbf{Internal reports} are reports that are needed either
by clients of the system or by contracts. For instance, the
\emph{ContractTemplates} report is needed by clients of the system in
order to instantiate contracts, and the \emph{Me} report is needed by
the two contracts, as we shall see in the following subsection.  A
list of internal reports, including a short description of what they
compute, is summarised in Figure~\ref{fig:internal-reports}. Except
for the \emph{Me} report, all internal reports are independent from
the custom data model.
\begin{figure}[t]
  \centering\small
  \begin{tabularx}{\textwidth}{l|X}
    \textbf{Report} & \textbf{Result}\\
    \hline\hline
    \emph{Me} & The special \recordname{Me} entity.\\
    \emph{Entities} & A list of all non-deleted entities.\\
    \emph{EntitiesByType} & A list of all non-deleted entities of a
    given type.\\
    \emph{ReportNames} & A list of names of all non-deleted
    reports.\\
    \emph{ReportNamesByTags} & A list of names of all non-deleted
    reports whose tags contain a given set and do not contain
    another given set.\\
    \emph{ReportTags} & A list of all tags used by non-deleted
    reports.\\
    \emph{ContractTemplates} & A list of names of all non-deleted
    contract templates.\\
    \emph{ContractTemplatesByType} & A list of names of all
    non-deleted contract templates of a given type.\\
    \emph{Contracts} & A list of all non-deleted contract
    instances.\\
    \emph{ContractHistory} & A list of previous transactions for a
    given contract instance.\\
    \emph{ContractSummary} & A list of meta data for a given
    contract instance.
  \end{tabularx}
  \caption{Internal reports.}
  \label{fig:internal-reports}
\end{figure}

~

\noindent\textbf{External reports} are reports that are expected to be
rendered directly in clients of the system, but they may also be
invoked by contracts. The external reports in our use case are the
reports mentioned earlier, namely the income statement, the balance
sheet, the cash flow statement, the list of unpaid invoices, and the
VAT report. Moreover, we include reports for calulating the list of
items in the inventory, and the list of top-$n$ customers,
respectively. We include the inventory report below as an example:
\begin{lstlisting}[language=parrot,basicstyle=\small]
report : Inventory
report =
 let itemsSold' = map (\i -> i{quantity = 0 - i.quantity}) itemsSold
 in
 -- The available items is the list of received items minus the
 -- list of reserved or sold items
 Inventory{availableItems = normaliseItems (itemsReceived ++ itemsSold')}
\end{lstlisting}
The value \prt{itemsSold} is defined in the domain-specific
prelude, similar to the value \prt{itemsReceived}. But unlike
\prt{itemsReceived}, the computation takes into account that items can be
reserved but not yet delivered. Hence when we check that items
are in stock using the inventory report, we also take into account
that some items in the inventory may have been sold, and therefore cannot
be sold again.

The five standard reports are defined according to the specifications
given by Henglein et al.~\cite[Section~2.1]{henglein09jlap}, but for
simplicity we do not model fixed costs, depreciation, and fixed
assets. We do, however, model multiple currencies, exemplified via
Danish Kroner (\recordname{DKK}) and Euro (\recordname{EUR}). This
means that financial reports, such as \prt{IncomeStatement}, provide
lists of values of type \recordname{Money}---one for each currency
used.


\subsection{Contracts}
\label{sec:use-case-contracts}
Contract specifications are divided into prelude functions
(Appendix~\ref{app:muerp-contracts-prelude}), domain-specific prelude
functions (Appendix~\ref{app:muerp-contracts-dsprelude}), and contract
templates (Appendix~\ref{app:muerp-contracts-templates}).

~

\noindent\textbf{Prelude functions} are utility functions similar to
the report engine's prelude functions. They are independent from the
custom data model, and are automatically added to all POETS
instances. The prelude includes standard functions such as
\prt{filter}.

~

\noindent\textbf{Domain-specific prelude functions} are utility
functions that depend on the custom data model. The \pcsl{inStock}
function, for example, checks whether the items described in a list of
order lines are in stock, by querying the \emph{Inventory} report (we
assume that the item types are different for each line):
\begin{lstlisting}[language=pcsl,basicstyle=\small]
fun inStock lines =
  let inv = (reports.inventory ()).availableItems
  in
  all (\l -> any (\i -> (l.item).itemType == i.itemType &&
                              (l.item).quantity <= i.quantity) inv) lines
\end{lstlisting}

~

\noindent\textbf{Contract templates} describe the daily activities in
the company, and in our \muerp use case we only consider a purchase
contract and a sales contract. The purchase contract is presented
below:
\lstinputlisting[language=pcsl,basicstyle=\small]{muERP/contracts/Purchase.csl}

The contract describes a simple workflow, in which the vendor delivers
items, possibly followed by an invoice, which in turn is followed by a
bank transfer of the company.  Note how the \pcsl{me} parameter in the
contract template body refers to the value from the domain-specific
prelude, which in turn invokes the \emph{Me} report. Note also how the
\pcsl{payment} clause template is recursively defined in order to
accommodate for potentially different currencies. That is, the total
payment is split up in as many bank transfers as there are currencies
in the purchase.

The sales contract is presented below:
\lstinputlisting[language=pcsl,basicstyle=\small]{muERP/contracts/Sale.csl}

The contract describes a workflow, in which the company issues an
invoice to the customer---but only if the items on the invoice are in
stock. The issuing of invoice is followed by an immediate (within an
hour) payment by the customer to the company, and a delivery of goods
by the company within a week. Moreover, we also model the repair
guarantee mentioned in the introduction.

\subsection{Bootstrapping the System}
\label{sec:bootstrapping}

The previous subsections described the specification code for
\muerp. Since data definitions, report specifications, and contract
specifications are added to the system at run-time, \muerp is
instantiated by invoking the following sequence of services on an
initially empty POETS instance:
\begin{enumerate}
\item Add data definitions in Appendix~\ref{app:muerp-ontology} via
  \emph{addDataDefs}.
\item Create a designated \recordname{Me} entity via
  \emph{createEntity}.
\item Add report specifications via \emph{addReport}.
\item Add contract specifications via \emph{createTemplate}.
\end{enumerate}

Hence, the event log will, conceptually, have the form (we write the
value of the field \fieldname{internalTimeStamp} before each event):
\begin{itemize}
\item[$t_1$:] \lstinline[language=eventlog]!AddDataDefs{defs = "ResourceType is ..."}!
\item[$t_2$:] \lstinline[language=eventlog]!CreateEntity{ent = $e_1$, recordType = "Me", data = Me}!
\item[$t_3$:] \lstinline[language=eventlog]!CreateReport{name = "Me", description = "Returns the ...",!\\
\lstinline[language=eventlog]! code = "name: Me\n ...", tags = ["internal","entity"]}!\\
$\vdots$
\item[$t_i$:] \lstinline[language=eventlog]!CreateReport{name = "TopNCustomers", description = "A list ...",!\\
\lstinline[language=eventlog]! code = "name: TopNCustomers\n ...",!\\
\lstinline[language=eventlog]! tags = ["external","financial","crm"]}!
\item[$t_{i+1}$:] \lstinline[language=eventlog]!CreateContractDef{name = "Purchase", recordType = "Purchase",!\\
\lstinline[language=eventlog]! code = "name: purchase\n ...", description = "Set up ..."}!
\item[$t_{i+2}$:] \lstinline[language=eventlog]!CreateContractDef{name = "Sale", recordType = "Sale",!\\
\lstinline[language=eventlog]! code = "name: sale\n ...", description = "Set up a sale"}!
\end{itemize}
for some increasing timestamps $t_1 < t_2 < \ldots < t_{i+2}$. Note
that the entity value $e_1$ of the \recordname{CreateEntity} event is
automatically generated by the entity store, as described in
Section~\ref{sec:entity-store}.

Following these operations, the system is operational. That is,
\begin{inparaenum}[(i)]
  \item customers and vendors can be managed via \emph{createEntity},
    \emph{updateEntity}, and \emph{deleteEntity},
  \item contracts can be instantiated, updated, concluded, and
    inspected via \emph{createContract}, \emph{updateContract},
    \emph{concludeContract}, and \emph{getContract} respectively,
  \item transactions can be registered via \emph{registerTransaction},
    and
  \item reports can be queried via \emph{queryReport}.
\end{inparaenum}

For example, if a sale is initiated with a new customer John Doe,
starting at time $t$, then the following events will be added to the
event log:
\begin{itemize}
\item[$t_{i+3}$:] \lstinline[language=eventlog]!CreateEntity{ent = $e_2$, recordType = "Customer", data = Customer{!\\
\lstinline[language=eventlog]!  name = "John Doe", address = Address{!\\
\lstinline[language=eventlog]!   string = "Universitetsparken 1", country = Denmark}}}!
\item[$t_{i+4}$:] \lstinline[language=eventlog]!CreateContract{contractId = 0, contract = Sale{!\\
\lstinline[language=eventlog]! startDate = $t$, templateName = "sale", customer = $e_2$,!\\
\lstinline[language=eventlog]! orderLines = [OrderLine{!\\
\lstinline[language=eventlog]!   item = Item{itemType = Bicycle{model = "Avenue"}, quantity = 1.0},!\\
\lstinline[language=eventlog]!   unitPrice = Money{currency = DKK, amount = 4000.0},!\\
\lstinline[language=eventlog]!   vatPercentage = 25.0}]}}!
\end{itemize}
That is, first the customer entity is created, and then we can
instantiate a new sales contract. In this particular sale, one bicycle
of the model ``Avenue'' is sold at a unit price of 4000 DKK, with an
additional VAT of 25 percent. Note that the contract id 0 of the
\recordname{CreateContract} is automatically generated and that the
start time $t$ is explicitly given in the
\recordname{CreateContract}'s \fieldname{startDate} field independent
from the \recordname{internalTimeStamp} field.

Following the events above, if the contract is executed
successfully, events of type \recordname{IssueInvoice},
\recordname{Delivery}, and \recordname{Payment} will persisted in the
event log with appropriate values---in particular, the payment will be
5000 DKK.

%%% Local Variables: 
%%% mode: latex
%%% TeX-master: "tr"
%%% End: 
